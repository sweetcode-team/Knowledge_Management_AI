\documentclass{article}

% Language setting
% Replace `english' with e.g. `spanish' to change the document language
\usepackage[italian]{babel}

% Set page size and margins
\usepackage[a4paper,top=3cm,bottom=3cm,left=3cm,right=3cm,marginparwidth=1.75cm]{geometry}

% Useful packages
% \usepackage{showframe}
% \usepackage{layout}
%comando per più righe su una cella di tabella
\newcommand{\quantities}[1]{%
  \begin{tabular}{@{}c@{}}\strut#1\strut\end{tabular}%
}

\usepackage[dvipsnames, table]{xcolor}
\usepackage{amsmath}
\usepackage{graphicx}
\usepackage[colorlinks=true, allcolors=blue]{hyperref}
\usepackage{tikz}
\usetikzlibrary{shapes, backgrounds, mindmap, trees}
\usepackage{fancyhdr}
\usetikzlibrary{positioning}
\usepackage[inkscapeformat=png]{svg}

\usepackage{hyperref}
\usepackage{lastpage}
\usepackage{moresize}
\usepackage{paracol}
\usepackage{enumitem}
\usepackage{nicematrix}
\usepackage{tabularx}
\usepackage{parskip}
\usepackage{fontspec}
\usepackage{style}
\usepackage{float}
\usepackage{setspace}

\setmainfont{Poppins}[
    Path=./Poppins/,
    Extension = .ttf,
    UprightFont=*-Regular,
    BoldFont=*-Bold,
    ItalicFont=*-Italic,
    BoldItalicFont=*-BoldItalic
    ]

\title{Titolo}
\author{SWEetCode}

\begin{document}
% \layout

\begin{titlepage}
    \thispagestyle{empty}
    \begin{tikzpicture}[remember picture, overlay]
        % TRIANGOLI
        \draw[fill=secondarycolor, secondarycolor] (current page.north west) -- (current page.south west) -- (8.8, -28);
        \draw[fill=primarycolor, primarycolor] (-3, 5) -- (4, -13.6) -- (11, 5);

        % LOGO
        \node [xshift=-5cm, yshift=25cm] (logo) at (current page.south east) {\includesvg[width=6.5cm]{logo.svg}};

        % SWEETCODE - DATE
        \node [anchor=north east, align=right, xshift=-1.2cm, yshift=20.5cm, text=black] (sweetcode) at (current page.south east) {\fontsize{32pt}{36pt}\selectfont SWEetCode};
        \draw[line width=4pt, lightcol] ([xshift=-3cm, yshift=-0.37cm]sweetcode.south west) -- ([yshift=-0.37cm]sweetcode.south east);
        \node [anchor=north east, align=right, xshift=-1.2cm, yshift=18.7cm, text=black] (date) at (current page.south east){\fontsize{24pt}{24pt} \selectfont 2023-11-08};

        % NOME FILE
        \node [anchor=north east, text width=15cm, align=right, xshift=-1.2cm, yshift=17cm, text=black] (titolo) at (current page.south east){\fontsize{44pt}{44pt}\textbf{Lettera di\\presentazione}}; % andare a capo con \\

        % BOX DATI PARTECIPANTI
        \node[anchor=north east, xshift=-1.2cm, yshift=12cm, minimum width=8cm] (box) at (current page.south east){};

        % VERSIONE
        \node[anchor=north west, align=left] (dati1) at (box.north west) {\fontsize{16pt}{16pt}\selectfont \textbf{}};
        % \draw[line width=4pt, lightcol] (dati1.south west) -- ([xshift=8cm]dati1.south west);
        \node[anchor=north west, align=left] (dati11) at (dati1.south west)
        {\fontsize{14pt}{14pt}\selectfont};

        % COMPONENTI DEL GRUPPO
        \node[anchor=north west, yshift=-1cm, align=left] (dati2) at (dati11.north west) {\fontsize{16pt}{16pt}\selectfont \textbf{Componenti del gruppo}};
        \draw[line width=4pt, lightcol] (dati2.south west) -- ([xshift=8cm]dati2.south west);
        \node[anchor=north west, align=left] (dati21) at (dati2.south west)
        {\fontsize{14pt}{14pt}\selectfont Bresolin G.};
        \node[anchor=north west, yshift=-0.7cm, align=left] (dati22) at (dati2.south west)
        {\fontsize{14pt}{14pt}\selectfont Campese M.};
        \node[anchor=north west, yshift=-1.4cm, align=left] (dati23) at (dati2.south west)
        {\fontsize{14pt}{14pt}\selectfont Ciriolo I.};
        \node[anchor=north west, yshift=-2.1cm, align=left] (dati24) at (dati2.south west)
        {\fontsize{14pt}{14pt}\selectfont Dugo A.};
        \node[anchor=north west, yshift=-2.8cm, align=left] (dati25) at (dati2.south west)
        {\fontsize{14pt}{14pt}\selectfont Feltrin E.};
        \node[anchor=north west, yshift=-3.5cm, align=left] (dati26) at (dati2.south west)
        {\fontsize{14pt}{14pt}\selectfont Michelon R.};
        \node[anchor=north west, yshift=-4.2cm, align=left] (dati27) at (dati2.south west)
        {\fontsize{14pt}{14pt}\selectfont Orlandi G.};
        
        % UNIPD - SWE
        \node [xshift=4.4cm, yshift=2.3cm, draw, secondarycolor, text=white] (uni) at (current page.south west) {\fontsize{20pt}{20pt} \selectfont Università di Padova};
        \node [xshift=0.65cm, yshift=0.7cm, draw, secondarycolor, text=white, below=of uni] (corso) {\fontsize{20pt}{20pt}\selectfont Ingegneria del Software};

        % FIRMA
        % \draw[line width=4pt, lightcol] ([xshift=-1.2cm, yshift=1.8cm]current page.south east) -- ([xshift=-8cm, yshift=1.8cm]current page.south east);
        % \node[anchor=north west, xshift=12.9cm, yshift=1.45cm, align=left] at (current page.south west)
        % {\fontsize{13pt}{13pt}\selectfont L'Amministratore: Feltrin E.};
        
    \end{tikzpicture}
\end{titlepage}
%Registro in ordine dalla più recente alla meno recente!
{\renewcommand{\arraystretch}{1.5}
\section*{Registro delle versioni}
\begin{tabularx}{\textwidth}{c|c|c|c|X}
\textbf{Versione} & \textbf{Data} & \quantities{\textbf{Responsabile di}\\\textbf{stesura}}& \textbf{Revisore} & \quantities{\textbf{Dettaglio e}\\\textbf{motivazioni}} \\
\hline
v0.0.1(23) & $2023-11-13$ & Campese M. & Feltrin E. & Modifiche correttive di versione e registro delle modifiche. \\
\hline
v0.0.1(10) & $2023-10-27$ & \quantities{Ciriolo I. \\ Feltrin E.} & Campese M. &  Stesura completa.\\

\end{tabularx}}
\newpage
Il gruppo SWEetCode è lieto di presentare la propria candidatura per la realizzazione del progetto “Knowledge management AI”, gentilmente proposto dall’azienda AzzurroDigitale.

Per una panoramica delle nostre attività, il gruppo SWEetCode estende un invito cordiale a visitare il sito:

\begin{center} 
\href{https://sweetcode-team.github.io}{https://sweetcode-team.github.io} 
\end{center}
\vspace{0.3cm}

La documentazione dettagliata, che include la valutazione dei capitolati d'appalto ed i verbali delle riunioni interne ed esterne, è disponibile al seguente link:

\begin{center}
\href{https://github.com/sweetcode-team/Documentation}{https://github.com/sweetcode-team/Documentation}
\end{center}
\vspace{0.05cm}


In tale spazio si potrà trovare:
\begin{itemize}
    \item Il preventivo dei costi e l’impegno totale e individuale;
    \end{itemize}
    \begin{itemize}
        \item L’analisi dettagliata dei singoli capitolati e le motivazioni della nostra scelta;
        \end{itemize}
        \begin{itemize}
            \item I verbali delle riunioni esterne tenute i giorni:
            \begin{itemize}
        \item 2023-10-23: Zucchetti Spa;
        \item 2023-10-25: AzzurroDigitale;
        \item 2023-10-26: Ergon Informatica Srl.
    \end{itemize}
            \end{itemize}
    
    \begin{itemize}
        \item I verbali delle riunioni interne tenute i giorni:
         \begin{itemize}
        \item 2023-10-17;
        \item 2023-10-20;
        \item 2023-10-27.
    \end{itemize}
        \end{itemize}

\vspace{2em}
Il costo stimato del progetto è \textbf{12.845,00 €}.

La data di consegna è prevista non oltre il \textbf{2024-04-08}.

\vspace{1em}
Il team è composto dai seguenti membri:
\\
\\
% TABELLA (tenere anche arraystretch che aumenta il padding verticale)
{\renewcommand{\arraystretch}{1.5}
\begin{tabularx}{\textwidth}{X|X}
\textbf{Cognome e Nome} & \textbf{Matricola n°} \\
\hline
Bresolin Gianluca & 2034316 \\
\hline
Campese Martina & 1122439 \\
\hline
Ciriolo Irene & 2043682 \\
\hline
Dugo Alberto & 2042382 \\
\hline
Feltrin Emanuele & 2034314 \\
\hline
Michelon Riccardo & 2042341 \\
\hline
Orlandi Giacomo & 1122253 \\
\end{tabularx}}\\
\\
\\
Con fiducia e sicurezza nella nostra competenza e nel nostro impegno, attendiamo con ansia l’opportunità di collaborare per la realizzazione di questo entusiasmante progetto.
\\
\\
Cordiali saluti,

Il gruppo SWEetCode
\end{document}