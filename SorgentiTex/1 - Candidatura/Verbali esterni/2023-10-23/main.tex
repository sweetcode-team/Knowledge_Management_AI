\documentclass{article}

% Language setting
% Replace `english' with e.g. `spanish' to change the document language
\usepackage[italian]{babel}

% Set page size and margins
\usepackage[a4paper,top=3cm,bottom=3cm,left=3cm,right=3cm,marginparwidth=1.75cm]{geometry}

% Useful packages
% \usepackage{showframe}
% \usepackage{layout}

\usepackage[dvipsnames, table]{xcolor}
\usepackage{amsmath}
\usepackage{graphicx}
\usepackage[colorlinks=true, allcolors=blue]{hyperref}
\usepackage{tikz}
\usetikzlibrary{shapes, backgrounds, mindmap, trees}
\usepackage{fancyhdr}
\usetikzlibrary{positioning}
\usepackage[inkscapeformat=png]{svg}

\usepackage{hyperref}
\usepackage{lastpage}
\usepackage{moresize}
\usepackage{paracol}
\usepackage{enumitem}
\usepackage{nicematrix}
\usepackage{tabularx}
\usepackage{parskip}
\usepackage{fontspec}
\usepackage{style}
\usepackage{float}
\usepackage{setspace}

\setmainfont{Poppins}[
    Path=./Poppins/,
    Extension = .ttf,
    UprightFont=*-Regular,
    BoldFont=*-Bold,
    ItalicFont=*-Italic,
    BoldItalicFont=*-BoldItalic
    ]

\title{Titolo}
\author{SWEetCode}

\begin{document}
% \layout

\begin{titlepage}
    \thispagestyle{empty}
    \begin{tikzpicture}[remember picture, overlay]
        % TRIANGOLI
        \draw[fill=secondarycolor, secondarycolor] (current page.north west) -- (current page.south west) -- (8.8, -28);
        \draw[fill=primarycolor, primarycolor] (-3, 5) -- (4, -13.6) -- (11, 5);

        % LOGO
        \node [xshift=-5cm, yshift=25cm] (logo) at (current page.south east) {\includesvg[width=6.5cm]{logo.svg}};

        % SWEETCODE - DATE
        \node [anchor=north east, align=right, xshift=-1.2cm, yshift=20.5cm, text=black] (sweetcode) at (current page.south east) {\fontsize{32pt}{36pt}\selectfont SWEetCode};
        \draw[line width=4pt, lightcol] ([xshift=-3cm, yshift=-0.37cm]sweetcode.south west) -- ([yshift=-0.37cm]sweetcode.south east);
        \node [anchor=north east, align=right, xshift=-1.2cm, yshift=18.7cm, text=black] (date) at (current page.south east){\fontsize{24pt}{24pt} \selectfont Verbale Esterno};% O ESTERNO

        % NOME FILE
        \node [anchor=north east, text width=15cm, align=right, xshift=-1.2cm, yshift=17cm, text=black] (titolo) at (current page.south east){\fontsize{48pt}{48pt}\textbf{2023-10-23}};

        % BOX DATI PARTECIPANTI
                \node[anchor=north east, xshift=-1.2cm, yshift=14.5cm, minimum width=8cm] (box) at (current page.south east){};
        
        % BOX DATI PARTECIPANTI
        %\node[anchor=north east, xshift=-1.2cm, yshift=12.5cm, minimum width=8cm] (box) at (current page.south east){};

        % AMMINISTRATORE
        %\node[anchor=north west, align=left] (dati1) at (box.north west) {\fontsize{15pt}{15pt}\selectfont \textbf{Amministratore}};
       % \draw[line width=4pt, lightcol] (dati1.south west) -- ([xshift=8cm]dati1.south west);
       % \node[anchor=north west, align=left] (dati11) at (dati1.south west)
       % {\fontsize{13pt}{13pt}\selectfont Feltrin E.};

        % RESPONSABILE
        \node[anchor=north west, align=left] (dati2) at  (box.north west) 
        {\fontsize{15pt}{15pt}\selectfont \textbf{Responsabile}};
        \draw[line width=4pt, lightcol] (dati2.south west) -- ([xshift=8cm]dati2.south west);
        \node[anchor=north west, align=left] (dati21) at (dati2.south west)
        {\fontsize{13pt}{13pt}\selectfont Feltrin E.};

        % VERIFICATORE
        \node[anchor=north west, yshift=-1cm, align=left] (dati3) at (dati21.north west) {\fontsize{15pt}{15pt}\selectfont \textbf{Verificatore}};
        \draw[line width=4pt, lightcol] (dati3.south west) -- ([xshift=8cm]dati3.south west);
        \node[anchor=north west, align=left] (dati31) at (dati3.south west)
        {\fontsize{14pt}{14pt}\selectfont Michelon R.};

        % SEGRETARIO DI RIUNIONE
        \node[anchor=north west, yshift=-1cm, align=left] (dati4) at (dati31.north west) {\fontsize{15pt}{15pt}\selectfont \textbf{Segretario di Riunione}};
        \draw[line width=4pt, lightcol] (dati4.south west) -- ([xshift=8cm]dati4.south west);
        \node[anchor=north west, align=left] (dati41) at (dati4.south west)
        {\fontsize{13pt}{13pt}\selectfont Ciriolo I.};

        % PARTECIPANTI
        %\node[anchor=north west, yshift=-1cm, align=left] (dati5) at (dati41.north west) {\fontsize{15pt}{15pt}\selectfont \textbf{Partecipanti}};
        %\draw[line width=4pt, lightcol] (dati5.south west) -- ([xshift=8cm]dati5.south west);
        %\node[anchor=north west, align=left] (dati51) at (dati5.south west)
        %{\fontsize{13pt}{13pt}\selectfont Campese M.};
        %\node[anchor=north west, yshift=-0.7cm, align=left] (dati52) at (dati5.south west)
        %{\fontsize{13pt}{13pt}\selectfont Ciriolo I.};
        %\node[anchor=north west, yshift=-1.4cm, align=left] (dati53) at (dati5.south west)
        %{\fontsize{13pt}{13pt}\selectfont Dugo A.};
        %\node[anchor=north west, yshift=-2.1cm, align=left] (dati54) at (dati5.south west)
        %{\fontsize{13pt}{13pt}\selectfont Michelon R.};
        
        % UNIPD - SWE
        \node [xshift=4.4cm, yshift=2.3cm, draw, secondarycolor, text=white] (uni) at (current page.south west) {\fontsize{20pt}{20pt} \selectfont Università di Padova};
        \node [xshift=0.65cm, yshift=0.7cm, draw, secondarycolor, text=white, below=of uni] (corso) {\fontsize{20pt}{20pt}\selectfont Ingegneria del Software};

        % FIRMA AZIENDA
        \draw[line width=4pt, lightcol] ([xshift=-1.2cm, yshift=5.2cm]current page.south east) -- ([xshift=-8cm, yshift=5.2cm]current page.south east);
        \node [xshift=-4.8cm, yshift=5.4cm] (logo) at (current page.south east) {\includegraphics[width=6cm]{firme/zucchetti_firma.png}};
        \node [xshift=-4.8cm, yshift=6.4cm] (logo) at (current page.south east) {\includegraphics[width=6cm]{firme/zucchetti_timbro.png}};
        \node[anchor=north west, xshift=12.9cm, yshift=4.95cm, align=left] at (current page.south west)
        {\fontsize{13pt}{13pt}\selectfont L'Azienda: Piccoli G.};

        % FIRMA
        \draw[line width=4pt, lightcol] ([xshift=-1.2cm, yshift=1.8cm]current page.south east) -- ([xshift=-8cm, yshift=1.8cm]current page.south east);
        \node [xshift=-4.8cm, yshift=2.45cm] (logo) at (current page.south east) {\includegraphics[width=6cm]{firme/emanuele.png}};
        \node[anchor=north west, xshift=12.9cm, yshift=1.45cm, align=left] at (current page.south west)
        {\fontsize{13pt}{13pt}\selectfont Il Responsabile: Feltrin E.};
        
    \end{tikzpicture}
\end{titlepage}

% INIZIO PAGINE
% use to vertically center content
% credits to: http://tex.stackexchange.com/questions/7219/how-to-vertically-center-two-images-next-to-each-other
\newcommand{\vcenteredinclude}[1]{\begingroup
\setbox0=\hbox{\includegraphics{#1}}%
\parbox{\wd0}{\box0}\endgroup}

% use to vertically center content
% credits to: http://tex.stackexchange.com/questions/7219/how-to-vertically-center-two-images-next-to-each-other
\newcommand*{\vcenteredhbox}[1]{\begingroup
\setbox0=\hbox{#1}\parbox{\wd0}{\box0}\endgroup}

\newcommand{\heading}[1] {
	\vspace{15pt}
	{\bf\LARGE\color{secondarycolor}\uppercase{#1}}\\[-4pt]
	{\color{primarycolor}\rule{0.1\textwidth}{2pt}}\vspace{2pt}
}

%---------------------------------------------------------
% New 
%----------------------------------------------------------
\newcommand{\titlebox}[3]
{\fcolorbox{#1}{#2}{\begin{minipage}[c][4.5cm][c]{\linewidth}%
\begin{center}\large\color{white} #3 %
\end{center}\end{minipage}\\[14pt]
\vspace{-12pt}
}
}

\newcommand{\bigfont}[1]{%
{\bf\huge\uppercase{#1} } \\[4pt]%
\rule{0.1\textwidth}{1.25pt} \\[4pt]%
}

\newcommand{\titletext}[1]{%
#1 \\[4pt] %
\rule{0.1\textwidth}{1.25pt} \\[4pt]%
}
           



\setlength{\parindent}{0mm}

\setlist[itemize]{label=\color{primarycolor}\textbullet}

%============================================================================%
\columnratio{0.31}
\setlength{\columnsep}{2.2em}
\setlength{\columnseprule}{4pt}
\colseprulecolor{lightcol}
\begin{paracol}{2}

\heading{Intestazione}

\textbf{Data} \\
2023-10-23\\

\textbf{Ora Inizio} \\
16:00\\

\textbf{Ora Fine} \\
17:00\\

\textbf{Luogo} \\
Piattaforma Zoom

\vspace{12.6em}

\heading{Partecipanti}

\textbf{Interni} \\
Bresolin G.\\
Campese M.\\
Ciriolo I.\\
Dugo A.\\
Feltrin E.\\
Michelon R.\\
Orlandi G.\\

\textbf{Esterni} \\
Piccoli G.\\


\switchcolumn

%---------------------------------------------------------------------------------------


\heading{Revisione delle Azioni}
\begin{enumerate}
Non vi sono state revisioni delle azioni.
\end{enumerate}
%\begin{enumerate}
    %\item 
   % \textbf{Azione 1:} suggests you ‘escape’ and save time by being unavailable (ruthlessly avoid going to useless meetings, etc.); you see what really matters, make time for (serious) play, get enough sleep and select what you spend your time with using ‘extreme criteria’. 
   % \item \textbf{Azione 2:} suggests you clarify decision making, dare to say no and learn how to do it gracefully without offending people, uncommit from non-essentials and gain freedom by setting boundaries for carefully ‘edited’ amount of meaningful activities.
   % \item \textbf{Azione 3:} praises using a “time buffer” between commitments, removing things which hurt your effectiveness most rather than starting some new quick fix technique on top of everything, progressing with small wins, using routine to get in the flow, focusing and being in the moment by asking (“What’s important now?”).
%\end{enumerate}

\vspace{26.3em}

\heading{Ordine del Giorno}
\begin{enumerate} 
    \item Decisione ruoli riunione;
    \item Discussione di alcune domande da parte del gruppo SWEetCode elaborate dai membri Dugo A., Michelon R., e Ciriolo I., e poste dal Responsabile Feltrin E. \\ All’incontro ha preso parte il rappresentante aziendale Piccoli G. da parte di Zucchetti Spa, per il capitolato numero C9-\textit{ChatSQL}.
   % \item \textbf{Parte 2:} suggests you clarify decision making, dare to say no and learn how to do it gracefully without offending people, uncommit from non-essentials and gain freedom by setting boundaries for carefully ‘edited’ amount of meaningful activities.
   % \item \textbf{Parte 3:} praises using a “time buffer” between commitments, removing things which hurt your effectiveness most rather than starting some new quick fix technique on top of everything, progressing with small wins, using routine to get in the flow, focusing and being in the moment by asking (“What’s important now?”).
\end{enumerate}

\newpage

\switchcolumn
\newpage

\heading{Discussione}\\
\textbf{Sintesi degli argomenti\\discussi\\}


\vspace{40cm}

\heading{Decisioni}\\
\textbf{Decisioni da \\intraprendere\\ successivamente alla \\discussione}

\switchcolumn
\begin{enumerate}
    \item 
    \textbf{L’azienda propone dei consigli sul modo in cui diverse parti di un'applicazione possono essere integrate in un unico Sistema?\\} È stato chiarito che l’applicazione sarà composta da due differenti interfacce: La prima utile al caricamento del dizionario dei dati, ed una seconda sviluppata per la produzione dei prompt dati dall’utente. Entrambe le interfacce faranno parte quindi di un'unica applicazione;
    \item \textbf{L'azienda consiglia un metodo o un framework per lo svolgimento del prgetto?\\} È stato da subito chiarito che le metodologie preferite dall’azienda sono \textit{Scrum} e \textit{Kanban}, ma non sono vincolanti; 
    \item \textbf{Esistono delle aspettative o particolari modelli di riferimento per la possibile interfaccia Utente (UI)?\\}L’azienda ha esposto una propensione verso un’interfaccia simile a quelle dei modelli più celebri (come l’interfaccia utente di \textit{Chat-Gpt}); Durante la discussione sono stati illustrati alcuni esempi di tali modelli per poterne prendere spunto (i link ai modelli si trovano nella sezione “Altro”);
    \item \textbf{Il database dovrà rispettare delle determinate specifiche tecniche?\\}Viene lasciata libertà totale per il formato del database: non è stato specificato nessun tipo di esempio e la possibile decisione viene delegata al gruppo ed alla sua progettazione;
    \item \textbf{L’azienda usa o consiglia determinate tecnologie per riassumere il dizionario dei dati nel prompt?\\} Anche in questo caso è stato chiarito che il gruppo sarà libero di scegliere il metodo migliore per l'ottimizzazione del prompt, a patto che risulti ragionevole per l'azienda proponente;
    \item \textbf{I requisiti dati dal documento di Presentazione del Capitolato sono rinegoziabili?\\} L’azienda ha confermato che trattandosi un progetto di ricerca è assolutamente possibile rinegoziare i requisiti;
    \item \textbf{L’azienda propone un modello di AI di riferimento o consigliato per lo svolgimento del progetto?\\} L’azienda lascia volutamente carta bianca per dare l’opportunità al gruppo di sperimentare diversi modelli e tecnologie, possibilmente evitando modelli richiedenti abbonamenti (poichè risulterebbero troppo costosi per le piccole imprese). L’azienda infatti ha espresso la sua preferenza per un modello gratuito come quelli citati in precedenza piuttosto che uno ad abbonamento.
    






\end{enumerate}

\vspace{7em}

\begin{enumerate}
    \item 
    Valutazione accurata del progetto \textit{ChatSQL} ed analisi dei punti a favore e contro alla luce delle nuove informazioni fornite dall’azienda proponente, con conseguente scelta del capitolato di preferenza del team.
   % \item \textbf{Part III (“Eliminate”)} suggests you clarify decision making, dare to say no and learn how to do it gracefully without offending people, uncommit from non-essentials and gain freedom by setting boundaries for carefully ‘edited’ amount of meaningful activities.
   % \item \textbf{Part IV (“Execute”)} praises using a “time buffer” between commitments, removing things which hurt your effectiveness most rather than starting some new quick fix technique on top of everything, progressing with small wins, using routine to get in the flow, focusing and being in the moment by asking (“What’s important now?”).
\end{enumerate}

\end{paracol}

\vspace{3cm}

\heading{Azioni da Intraprendere}

{\renewcommand{\arraystretch}{1.5}
\begin{tabularx}{\textwidth}{X|c|c|c}
\textbf{Azione} & \textbf{Incaricato} & \textbf{Revisore} & \textbf{Scadenza} \\
\hline
Stesura del verbale della riunione & Ciriolo I. & Michelon R. & $2023-10-28$ \\

%Setup workspace informale & Dugo A. & Team & $2023-10-17$ \\
%\hline
%Creazione template contenuti verbali & Bresolin G. & Team & $2023-10-20$ \\
%\hline
%Stesura verbale interno 2023-10-17 & Bresolin G. & Feltrin E. & $2023-10-20$ \\
%\hline
%Creazione logo & Michelon R. & Team & $2023-10-20$ \\
\end{tabularx}}

\vspace{3em}

\heading{Altro}

\textbf{Materiale condiviso durante la riunione\\}
Siti web per la consultazione di modelli:
\begin{itemize}
    \item \href{https://huggingface.co/}{Hugging Face} 
\end{itemize}
\begin{itemize}
    \item \href{https://chat.lmsys.org/}{Chat.lmsys}\\
\end{itemize}

\textbf{Prossima Riunione}\\
Non sono previste riunioni successive con i rappresentanti dell'azienda Zucchetti Spa.
\end{document}