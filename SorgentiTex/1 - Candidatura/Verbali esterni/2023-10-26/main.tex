\documentclass{article}

% Language setting
% Replace `english' with e.g. `spanish' to change the document language
\usepackage[italian]{babel}

% Set page size and margins
\usepackage[a4paper,top=3cm,bottom=3cm,left=3cm,right=3cm,marginparwidth=1.75cm]{geometry}

% Useful packages
% \usepackage{showframe}
% \usepackage{layout}

\usepackage[dvipsnames, table]{xcolor}
\usepackage{amsmath}
\usepackage{graphicx}
\usepackage[colorlinks=true, allcolors=blue]{hyperref}
\usepackage{tikz}
\usetikzlibrary{shapes, backgrounds, mindmap, trees}
\usepackage{fancyhdr}
\usetikzlibrary{positioning}
\usepackage[inkscapeformat=png]{svg}
\usepackage{hyperref}
\usepackage{lastpage}
\usepackage{moresize}
\usepackage{paracol}
\usepackage{enumitem}
\usepackage{nicematrix}
\usepackage{tabularx}
\usepackage{parskip}
\usepackage{fontspec}
\usepackage{style}
\usepackage{float}
\usepackage{setspace}

\setmainfont{Poppins}[
    Path=./Poppins/,
    Extension = .ttf,
    UprightFont=*-Regular,
    BoldFont=*-Bold,
    ItalicFont=*-Italic,
    BoldItalicFont=*-BoldItalic
    ]

\title{Titolo}
\author{SWEetCode}

\begin{document}
% \layout

\begin{titlepage}
    \thispagestyle{empty}
    \begin{tikzpicture}[remember picture, overlay]
        % TRIANGOLI
        \draw[fill=secondarycolor, secondarycolor] (current page.north west) -- (current page.south west) -- (8.8, -28);
        \draw[fill=primarycolor, primarycolor] (-3, 5) -- (4, -13.6) -- (11, 5);

        % LOGO
        \node [xshift=-5cm, yshift=25cm] (logo) at (current page.south east) {\includesvg[width=6.5cm]{logo.svg}};
        
        % SWEETCODE - DATE
        \node [anchor=north east, align=right, xshift=-1.2cm, yshift=20.5cm, text=black] (sweetcode) at (current page.south east) {\fontsize{32pt}{36pt}\selectfont SWEetCode};
        \draw[line width=4pt, lightcol] ([xshift=-3cm, yshift=-0.37cm]sweetcode.south west) -- ([yshift=-0.37cm]sweetcode.south east);
        \node [anchor=north east, align=right, xshift=-1.2cm, yshift=18.7cm, text=black] (date) at (current page.south east){\fontsize{24pt}{24pt} \selectfont Verbale Esterno};% O ESTERNO

        % NOME FILE
        \node [anchor=north east, text width=15cm, align=right, xshift=-1.2cm, yshift=17cm, text=black] (2023-10-17) at (current page.south east){\fontsize{48pt}{48pt}\textbf{2023-10-26}};

        % BOX DATI PARTECIPANTI
                \node[anchor=north east, xshift=-1.2cm, yshift=14.5cm, minimum width=8cm] (box) at (current page.south east){};


        % RESPONSABILE
        \node[anchor=north west, yshift=-1cm, align=left] (dati2) at (box.north west) {\fontsize{15pt}{15pt}\selectfont \textbf{Responsabile}};
        \draw[line width=4pt, lightcol] (dati2.south west) -- ([xshift=8cm]dati2.south west);
        \node[anchor=north west, align=left] (dati21) at (dati2.south west)
        {\fontsize{13pt}{13pt}\selectfont Feltrin E.};

        % VERIFICATORE
        \node[anchor=north west, yshift=-1cm, align=left] (dati3) at (dati21.north west) {\fontsize{15pt}{15pt}\selectfont \textbf{Verificatore}};
        \draw[line width=4pt, lightcol] (dati3.south west) -- ([xshift=8cm]dati3.south west);
        \node[anchor=north west, align=left] (dati31) at (dati3.south west)
        {\fontsize{13pt}{13pt}\selectfont Campese M.};

        % SEGRETARIO DI RIUNIONE
        \node[anchor=north west, yshift=-1cm, align=left] (dati4) at (dati31.north west) {\fontsize{15pt}{15pt}\selectfont \textbf{Segretario di Riunione}};
        \draw[line width=4pt, lightcol] (dati4.south west) -- ([xshift=8cm]dati4.south west);
        \node[anchor=north west, align=left] (dati41) at (dati4.south west)
        {\fontsize{13pt}{13pt}\selectfont Dugo A.};
        
        % UNIPD - SWE
        \node [xshift=4.4cm, yshift=2.3cm, draw, secondarycolor, text=white] (uni) at (current page.south west) {\fontsize{20pt}{20pt} \selectfont Università di Padova};
        \node [xshift=0.65cm, yshift=0.7cm, draw, secondarycolor, text=white, below=of uni] (corso) {\fontsize{20pt}{20pt}\selectfont Ingegneria del Software};
        
        % FIRMA AZIENDA
        \draw[line width=4pt, lightcol] ([xshift=-1.2cm, yshift=5.2cm]current page.south east) -- ([xshift=-8cm, yshift=5.2cm]current page.south east);
        \node [xshift=-4.8cm, yshift=6cm] (logo) at (current page.south east) {\includegraphics[width=6cm]{firme/ergon.png}};
        \node[anchor=north west, xshift=12.9cm, yshift=4.95cm, align=left] at (current page.south west)
        {\fontsize{13pt}{13pt}\selectfont L'azienda: Carlesso G.};
        
        % FIRMA INTERNA
        \draw[line width=4pt, lightcol] ([xshift=-1.2cm, yshift=1.8cm]current page.south east) -- ([xshift=-8cm, yshift=1.8cm]current page.south east);
        \node [xshift=-4.8cm, yshift=2.45cm] (logo) at (current page.south east) {\includegraphics[width=6cm]{firme/emanuele.png}};
        \node[anchor=north west, xshift=12.9cm, yshift=1.45cm, align=left] at (current page.south west)
        {\fontsize{13pt}{13pt}\selectfont Il Responsabile: Feltrin E.};
        
    \end{tikzpicture}
\end{titlepage}

% INIZIO PAGINE

\setlength{\parindent}{0mm}

\setlist[itemize]{label=\color{primarycolor}\textbullet}

% use to vertically center content
% credits to: http://tex.stackexchange.com/questions/7219/how-to-vertically-center-two-images-next-to-each-other
\newcommand{\vcenteredinclude}[1]{\begingroup
\setbox0=\hbox{\includegraphics{#1}}%
\parbox{\wd0}{\box0}\endgroup}

% use to vertically center content
% credits to: http://tex.stackexchange.com/questions/7219/how-to-vertically-center-two-images-next-to-each-other
\newcommand*{\vcenteredhbox}[1]{\begingroup
\setbox0=\hbox{#1}\parbox{\wd0}{\box0}\endgroup}

\newcommand{\heading}[1] {
	\vspace{15pt}
	{\bf\LARGE\color{secondarycolor}\uppercase{#1}}\\[-4pt]
	{\color{primarycolor}\rule{0.1\textwidth}{2pt}}\vspace{2pt}
}

%---------------------------------------------------------
% New 
%----------------------------------------------------------
\newcommand{\titlebox}[3]
{\fcolorbox{#1}{#2}{\begin{minipage}[c][4.5cm][c]{\linewidth}%
\begin{center}\large\color{white} #3 %
\end{center}\end{minipage}\\[14pt]
\vspace{-12pt}
}
}

\newcommand{\bigfont}[1]{%
{\bf\huge\uppercase{#1} } \\[4pt]%
\rule{0.1\textwidth}{1.25pt} \\[4pt]%
}

\newcommand{\titletext}[1]{%
#1 \\[4pt] %
\rule{0.1\textwidth}{1.25pt} \\[4pt]%
}
           




%============================================================================%
\columnratio{0.31}
\setlength{\columnsep}{2.2em}
\setlength{\columnseprule}{4pt}
\colseprulecolor{lightcol}
\begin{paracol}{2}

\heading{Intestazione}

\textbf{Data} \\
2023-10-26\\

\textbf{Ora Inizio} \\
17:00\\

\textbf{Ora Fine} \\
17:30\\

\textbf{Luogo} \\
Zoom

\vspace{6em}

\heading{Partecipanti}

\textbf{Interni} \\
Campese M.\\
Dugo A.\\
Feltrin E.\\
Michelon R.\\
Orlandi G.\\

\textbf{Esterni} \\
Carlesso G.\\

\textbf{Assenti} \\
Bresolin G.\\
Ciriolo I.

\newpage

\heading{Discussione}\\
\textbf{Sintesi degli argomenti\\discussi}

\newpage
~ \\\\\\\\\\\\\\\\\\\\\\\\\\\\\\\\\\\\\\\\\\~

\heading{Decisioni}\\
\textbf{Decisioni da\\intraprendere\\successivamente alla\\discussione}
% fine colonna sinistra
\switchcolumn
%---------------------------------------------------------------------------------------


\heading{Revisione delle Azioni}
\begin{enumerate}
Non vi sono state revisioni delle azioni.
\end{enumerate}
\vspace{19.8em}

\heading{Ordine del Giorno}\\
\begin{enumerate}
    \item Decisione ruoli riunione; 
    \item Approfondimento con \textit{Ergon} sul capitolato "Sistema di Raccomandazione" e sul \textit{Way of Working} da utilizzare.
\end{enumerate}
\newpage
\begin{enumerate}[i.]

    \item \textbf{Con quale frequenza avverrà il contatto con l’azienda?}\\
    Molto dipende dalla frequenza con la quale il gruppo “SWEetCode” vuole incontrarsi con Ergon, solitamente cadenza settimanale. In particolare molto dipende dalla metodologia di lavoro del gruppo, poiché alcuni gruppi preferiscono avere più incontri ed essere maggiormente seguiti, altri si trovano meglio ad avere solo pochi incontri;
    \item \textbf{Qual è la vostra metodologia di lavoro? Utilizzate un approccio \textit{Agile}, \textit{Scrum}, o un altro tipo di metodologia?}\\
    Ergon utilizza principalmente l'approccio \textit{Agile} il quale è consigliato che venga implementato durante lo sviluppo del progetto: “Sistemi di raccomandazione”;
    \item \textbf{Quale ambiente consigliate per il tracciamento delle segnalazioni? \textit{GitHub Issue? Jira? Notion? YouTrack?}}\\
    In Ergon, all'interno dell'azienda, utilizzano un software proprietario, ma con alcuni clienti prediligono l’utilizzo di \textit{Asana} e \textit{Trello};
    \item \textbf{Quali sono le specifiche dell’interfaccia utente?} \\
    L’azienda non ha specificato come implementare in modo specifico nel capitolato l’interfaccia utente perché vogliono lasciare spazio al gruppo;\\
    È possibile però effettuare una sessione di design thinking con i loro esperti nella quale si andrà a fare una bozza dell'interfaccia utente;
    \item \textbf{Quali "nice to have" potrebbe avere l’interfaccia utente?}\\
    Un "nice to have" potrebbe essere l'implementazione di un sistema di filtraggio;
    \item \textbf{Come verrà gestita la diversità dei prodotti nel sistema di raccomandazione? Ad esempio, come verranno gestiti i prodotti molto popolari rispetto a quelli meno conosciuti?}\\
    Molto dipende come andremo ad addestrare il modello, infatti dipenderà da quanto peso daremo alla feature legata alla quantità di volte in cui il prodotto è stato acquistato. Se gli daremo tanto peso all’interno del modello, allora esso sarà maggiormente biased da questa feature;  i clienti interessati a un determinato argomento quindi, riceveranno, nell’ordine dei prodotti, prima quelli maggiormente acquistati;
    \item \textbf{Come andremo a misurare le performance del sistema di raccomandazione durante il suo sviluppo?}\\
    Esse verranno fatte in base a delle metriche, in particolare, si andrà a utilizzare la tecnica \textit{Hold-out} la quale utilizza \textit{training set}, \textit{validation set} e \textit{test set} per verificare le performance e scegliere il miglior modello, in modo da avere maggiori probabilità che esso sia fair e che non sia affetto da bias;
    \item \textbf{Quando avremo bisogno di dati per addestrare il modello, sarete in grado di fornirci un \textit{dataset}? Se sì, quale tipo di \textit{dataset} sareste in grado di fornirci?}\\
    Ergon provvederà all’estrazione di un \textit{dataset} da dati reali relativi al settore alimentare e ci saranno forniti circa 2 milioni di feedback sui quali lavorare;
    \item \textbf{Come viene gestito il problema del cold start? Quando un nuovo utente si unisce al sistema, come vengono generate le raccomandazioni iniziali?}\\
    Questo è spesso il problema cruciale, si possono ricevere inizialmente feedback espliciti da parte dell’utente (ad esempio: preferenze e tipologie di interesse) per inizializzare il processo di raccomandazione, che verrà poi raffinato nel tempo tramite l’interazione dell’utente con il sistema.
\end{enumerate}
\vspace{1.6cm}

\begin{enumerate}
    \item Valutazione accurata del progetto ”Sistema di Raccomandazione” ed analisi dei punti a favore e contro alla luce delle nuove informazioni fornite dall’azienda proponente, con conseguente scelta del capito-lato di preferenza del team.
\end{enumerate}


\end{paracol}

\vspace{2cm}


\heading{Azioni da Intraprendere}


{\renewcommand{\arraystretch}{1.5}
\begin{tabularx}{\textwidth}{X|c|c|c}
\textbf{Azione} & \textbf{Incaricato} & \textbf{Revisore} & \textbf{Scadenza} \\
\hline
Stesura verbale esterno 2023-10-26 & Dugo A. & Campese M. & $2023-10-27$ \\
\end{tabularx}}

\vspace{2cm}

\heading{Altro}

\textbf{Prossima Riunione}

Non è stata fissata una prossima data per una nuova riunione tra SWEetCode e Ergon.

\end{document}