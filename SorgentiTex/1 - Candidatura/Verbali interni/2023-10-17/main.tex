\documentclass{article}

% Language setting
% Replace `english' with e.g. `spanish' to change the document language
\usepackage[italian]{babel}

% Set page size and margins
\usepackage[a4paper,top=3cm,bottom=3cm,left=3cm,right=3cm,marginparwidth=1.75cm]{geometry}

% Useful packages
% \usepackage{showframe}
% \usepackage{layout}

\usepackage[dvipsnames, table]{xcolor}
\usepackage{amsmath}
\usepackage{graphicx}
\usepackage[colorlinks=true, allcolors=blue]{hyperref}
\usepackage{tikz}
\usetikzlibrary{shapes, backgrounds, mindmap, trees}
\usepackage{fancyhdr}
\usetikzlibrary{positioning}
\usepackage[inkscapeformat=png]{svg}
\usepackage{hyperref}
\usepackage{lastpage}
\usepackage{moresize}
\usepackage{paracol}
\usepackage{enumitem}
\usepackage{nicematrix}
\usepackage{tabularx}
\usepackage{parskip}
\usepackage{fontspec}
\usepackage{style}
\usepackage{float}
\usepackage{setspace}

\setmainfont{Poppins}[
    Path=./Poppins/,
    Extension = .ttf,
    UprightFont=*-Regular,
    BoldFont=*-Bold,
    ItalicFont=*-Italic,
    BoldItalicFont=*-BoldItalic
    ]

\title{Titolo}
\author{SWEetCode}

\begin{document}
% \layout

\begin{titlepage}
    \thispagestyle{empty}
    \begin{tikzpicture}[remember picture, overlay]
        % TRIANGOLI
        \draw[fill=secondarycolor, secondarycolor] (current page.north west) -- (current page.south west) -- (8.8, -28);
        \draw[fill=primarycolor, primarycolor] (-3, 5) -- (4, -13.6) -- (11, 5);

        % LOGO
        \node [xshift=-5cm, yshift=25cm] (logo) at (current page.south east) {\includesvg[width=6.5cm]{logo.svg}};
        
        % SWEETCODE - DATE
        \node [anchor=north east, align=right, xshift=-1.2cm, yshift=20.5cm, text=black] (sweetcode) at (current page.south east) {\fontsize{32pt}{36pt}\selectfont SWEetCode};
        \draw[line width=4pt, lightcol] ([xshift=-3cm, yshift=-0.37cm]sweetcode.south west) -- ([yshift=-0.37cm]sweetcode.south east);
        \node [anchor=north east, align=right, xshift=-1.2cm, yshift=18.7cm, text=black] (date) at (current page.south east){\fontsize{24pt}{24pt} \selectfont Verbale Interno};% O ESTERNO

        % NOME FILE
        \node [anchor=north east, text width=15cm, align=right, xshift=-1.2cm, yshift=17cm, text=black] (2023-10-17) at (current page.south east){\fontsize{48pt}{48pt}\textbf{2023-10-17}};

        % BOX DATI PARTECIPANTI
        \node[anchor=north east, xshift=-1.2cm, yshift=12.5cm, minimum width=8cm] (box) at (current page.south east){};
        % RESPONSABILE
        \node[anchor=north west, align=left] (dati2) at (box.north west) {\fontsize{15pt}{15pt}\selectfont \textbf{Responsabile}};
        \draw[line width=4pt, lightcol] (dati2.south west) -- ([xshift=8cm]dati2.south west);
        \node[anchor=north west, align=left] (dati21) at (dati2.south west)
        {\fontsize{13pt}{13pt}\selectfont Feltrin E.};

        % VERIFICATORE
        \node[anchor=north west, yshift=-1cm, align=left] (dati3) at (dati21.north west) {\fontsize{15pt}{15pt}\selectfont \textbf{Verificatore}};
        \draw[line width=4pt, lightcol] (dati3.south west) -- ([xshift=8cm]dati3.south west);
        \node[anchor=north west, align=left] (dati31) at (dati3.south west)
        {\fontsize{13pt}{13pt}\selectfont Dugo A.};

        % SEGRETARIO DI RIUNIONE
        \node[anchor=north west, yshift=-1cm, align=left] (dati4) at (dati31.north west) {\fontsize{15pt}{15pt}\selectfont \textbf{Segretario di Riunione}};
        \draw[line width=4pt, lightcol] (dati4.south west) -- ([xshift=8cm]dati4.south west);
        \node[anchor=north west, align=left] (dati41) at (dati4.south west)
        {\fontsize{13pt}{13pt}\selectfont Bresolin G.};
        
        % UNIPD - SWE
        \node [xshift=4.4cm, yshift=2.3cm, draw, secondarycolor, text=white] (uni) at (current page.south west) {\fontsize{20pt}{20pt} \selectfont Università di Padova};
        \node [xshift=0.65cm, yshift=0.7cm, draw, secondarycolor, text=white, below=of uni] (corso) {\fontsize{20pt}{20pt}\selectfont Ingegneria del Software};

        % FIRMA
        \draw[line width=4pt, lightcol] ([xshift=-1.2cm, yshift=1.8cm]current page.south east) -- ([xshift=-8cm, yshift=1.8cm]current page.south east);
        \node [xshift=-4.8cm, yshift=2.45cm] (logo) at (current page.south east) {\includegraphics[width=6cm]{firme/emanuele.png}};
        \node[anchor=north west, xshift=12.9cm, yshift=1.45cm, align=left] at (current page.south west)
        {\fontsize{13pt}{13pt}\selectfont Il Responsabile: Feltrin E.};
        
    \end{tikzpicture}
\end{titlepage}

% INIZIO PAGINE

\setlength{\parindent}{0mm}

\setlist[itemize]{label=\color{primarycolor}\textbullet}

% use to vertically center content
% credits to: http://tex.stackexchange.com/questions/7219/how-to-vertically-center-two-images-next-to-each-other
\newcommand{\vcenteredinclude}[1]{\begingroup
\setbox0=\hbox{\includegraphics{#1}}%
\parbox{\wd0}{\box0}\endgroup}

% use to vertically center content
% credits to: http://tex.stackexchange.com/questions/7219/how-to-vertically-center-two-images-next-to-each-other
\newcommand*{\vcenteredhbox}[1]{\begingroup
\setbox0=\hbox{#1}\parbox{\wd0}{\box0}\endgroup}

\newcommand{\heading}[1] {
	\vspace{15pt}
	{\bf\LARGE\color{secondarycolor}\uppercase{#1}}\\[-4pt]
	{\color{primarycolor}\rule{0.1\textwidth}{2pt}}\vspace{2pt}
}

%---------------------------------------------------------
% New 
%----------------------------------------------------------
\newcommand{\titlebox}[3]
{\fcolorbox{#1}{#2}{\begin{minipage}[c][4.5cm][c]{\linewidth}%
\begin{center}\large\color{white} #3 %
\end{center}\end{minipage}\\[14pt]
\vspace{-12pt}
}
}

\newcommand{\bigfont}[1]{%
{\bf\huge\uppercase{#1} } \\[4pt]%
\rule{0.1\textwidth}{1.25pt} \\[4pt]%
}

\newcommand{\titletext}[1]{%
#1 \\[4pt] %
\rule{0.1\textwidth}{1.25pt} \\[4pt]%
}
           




%============================================================================%
\columnratio{0.31}
\setlength{\columnsep}{2.2em}
\setlength{\columnseprule}{4pt}
\colseprulecolor{lightcol}
\begin{paracol}{2}

\heading{Intestazione}

\textbf{Data} \\
2023-10-17\\

\textbf{Ora Inizio} \\
14:00\\

\textbf{Ora Fine} \\
16:30\\

\textbf{Luogo} \\
Aule Via Luzzati, LUM250

\vspace{6em}

\heading{Partecipanti}

\textbf{Interni} \\
Bresolin G.\\
Campese M.\\
Ciriolo I.\\
Dugo A.\\
Feltrin E.\\
Michelon R.\\
Orlandi G.\\

% colonna sinistra

\newpage

\heading{Discussione}\\
\textbf{Sintesi degli argomenti\\discussi}

\newpage

\heading{Decisioni}\\
\textbf{Decisioni prese durante\\la discussione}

% fine colonna sinistra
\switchcolumn
%---------------------------------------------------------------------------------------


\heading{Revisione delle Azioni}
\begin{enumerate}
Non vi sono state revisioni delle azioni.
\end{enumerate}

\vspace{19,8em}
\heading{Ordine del Giorno}

\begin{enumerate}[i.]
    \item Decisione ruoli riunione;
    \item Decisione nome gruppo;
    \item Logo gruppo;
    \item Analisi e scelta capitolati;
    \item Creazione email gruppo;
    \item Creazione account \textit{GitHub} gruppo;
    \item Creazione workspace informale;
    \item Attività future da svolgere.
\end{enumerate}

\newpage

\begin{enumerate}
    \item Assegnazione ruoli riunione:
    \begin{itemize}
        \item Responsabile: Feltrin E.;
        \item Verificatore: Dugo A.;
        \item Segretario di riunione: Bresolin G.
    \end{itemize}
    \item Decisione del nome ufficiale del gruppo a partire da un elenco fornito da Ciriolo Irene e Campese Martina, arrivando attraverso una votazione alla scelta del nome “SWEetCode”;
    \item Assegnazione compito "creazione logo" a Michelon Riccardo;
    \item Discussione e analisi in merito alle presentazioni dei capitolati presentati durante la lezione in data 2023-10-17 dalla quale è emerso interesse per i capitolati C-1, C-2 e C-9. Si è inoltre deciso di contattare le aziende AzzurroDigitale, Ergon e Zucchetti per richiedere un colloquio nel quale ottenere chiarimenti in merito al materiale da loro presentato;
    \item Creazione dell’account \textit{Gmail} del gruppo:
    sweetcode.team@gmail.com;
    \item Creazione di un account \textit{GitHub} privato e informale del gruppo utilizzando l'email precedentemente creata, che non necessita di essere analizzata dal docente;
    \item Dal gruppo è emersa la necessità di un workspace per la condivisione interna di materiale e documenti informali che non necessita di essere analizzata dal docente: la piattaforma scelta è \textit{Notion} ed il suo setup è stato affidato a Dugo Alberto, con successiva collaborazione da parte di tutti i membri del team;
    \item Dalla discussione sulle attività future da svolgere sono emerse:
    \begin{itemize}
        \item Approfondimento delle tecnologie da utilizzare nei capitolati di interesse da cui poter estrarre domande pertinenti e significative da porre alle aziende;
        \item Creazione di un template per il contenuto da utilizzare per la stesura dei verbali interni ed esterni.
    \end{itemize}
\end{enumerate}


\newpage


\textbf{Nome gruppo:} SWEetCode.

\textbf{Capitolati d'interesse (in ordine):}
\begin{enumerate}
    \item C-1: “Knowledge management AI”;
    \item C-9: “ChatSQL: creare frasi SQL da linguaggio naturale”;
    \item C-2: “Sistemi di raccomandazione”.
\end{enumerate}

\textbf{Email gruppo:} sweetcode.team@gmail.com.

\textbf{Account GitHub privato:} informazione volontariamente omessa.

\textbf{Piattaforma workspace informale:} Notion.

\end{paracol}

\vspace{3cm}


\heading{Azioni da Intraprendere}


{\renewcommand{\arraystretch}{1.5}
\begin{tabularx}{\textwidth}{X|c|c|c}
\textbf{Azione} & \textbf{Incaricato} & \textbf{Revisore} & \textbf{Scadenza} \\
\hline
Setup workspace informale & Dugo A. & Team & $2023-10-17$ \\
\hline
Contatto aziende & Ciriolo I. & Team & $2023-10-19$ \\
\hline
Creazione template contenuti verbali & Bresolin G. & Team & $2023-10-20$ \\
\hline
Stesura verbale interno 2023-10-17 & Bresolin G. & Feltrin E. & $2023-10-20$ \\
\hline
Creazione logo & Michelon R. & Team & $2023-10-20$ \\
\end{tabularx}}
\vspace{3cm}

\heading{Altro}

\textbf{Prossima Riunione}: 2023-10-20

\end{document}