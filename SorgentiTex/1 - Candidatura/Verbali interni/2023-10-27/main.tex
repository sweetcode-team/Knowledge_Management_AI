\documentclass{article}

% Language setting
% Replace `english' with e.g. `spanish' to change the document language
\usepackage[italian]{babel}

% Set page size and margins
\usepackage[a4paper,top=3cm,bottom=3cm,left=3cm,right=3cm,marginparwidth=1.75cm]{geometry}

%Comando per creare più righe nella stessa cella di una tabella ( si usa così: \quantities{martina \\ gianluca})
\newcommand{\quantities}[1]{%
  \begin{tabular}{@{}c@{}}\strut#1\strut\end{tabular}%
}
% Useful packages
% \usepackage{showframe}
% \usepackage{layout}

\usepackage[dvipsnames, table]{xcolor}
\usepackage{amsmath}
\usepackage{graphicx}
\usepackage[colorlinks=true, allcolors=blue]{hyperref}
\usepackage{tikz}
\usetikzlibrary{shapes, backgrounds, mindmap, trees}
\usepackage{fancyhdr}
\usetikzlibrary{positioning}
\usepackage[inkscapeformat=png]{svg}

\usepackage{hyperref}
\usepackage{lastpage}
\usepackage{moresize}
\usepackage{paracol}
\usepackage{enumitem}
\usepackage{nicematrix}
\usepackage{tabularx}
\usepackage{parskip}
\usepackage{fontspec}
\usepackage{style}
\usepackage{float}
\usepackage{setspace}

\setmainfont{Poppins}[
    Path=./Poppins/,
    Extension = .ttf,
    UprightFont=*-Regular,
    BoldFont=*-Bold,
    ItalicFont=*-Italic,
    BoldItalicFont=*-BoldItalic
    ]

\title{Titolo}
\author{SWEetCode}

\begin{document}
% \layout

\begin{titlepage}
    \thispagestyle{empty}
    \begin{tikzpicture}[remember picture, overlay]
        % TRIANGOLI
        \draw[fill=secondarycolor, secondarycolor] (current page.north west) -- (current page.south west) -- (8.8, -28);
        \draw[fill=primarycolor, primarycolor] (-3, 5) -- (4, -13.6) -- (11, 5);

        % LOGO
        \node [xshift=-5cm, yshift=25cm] (logo) at (current page.south east) {\includesvg[width=6.5cm]{logo.svg}};

        % SWEETCODE - DATE
        \node [anchor=north east, align=right, xshift=-1.2cm, yshift=20.5cm, text=black] (sweetcode) at (current page.south east) {\fontsize{32pt}{36pt}\selectfont SWEetCode};
        \draw[line width=4pt, lightcol] ([xshift=-3cm, yshift=-0.37cm]sweetcode.south west) -- ([yshift=-0.37cm]sweetcode.south east);
        \node [anchor=north east, align=right, xshift=-1.2cm, yshift=18.7cm, text=black] (date) at (current page.south east){\fontsize{24pt}{24pt} \selectfont Verbale Interno};% O ESTERNO

        % NOME FILE
        \node [anchor=north east, text width=15cm, align=right, xshift=-1.2cm, yshift=17cm, text=black] (titolo) at (current page.south east){\fontsize{48pt}{48pt}\textbf{2023-10-27}};

        % BOX DATI PARTECIPANTI
        \node[anchor=north east, xshift=-1.2cm, yshift=12.5cm, minimum width=8cm] (box) at (current page.south east){};

        % RESPONSABILE
        \node[anchor=north west, align=left] (dati2) at (box.north west) {\fontsize{15pt}{15pt}\selectfont \textbf{Responsabile}};
        \draw[line width=4pt, lightcol] (dati2.south west) -- ([xshift=8cm]dati2.south west);
        \node[anchor=north west, align=left] (dati21) at (dati2.south west)
        {\fontsize{13pt}{13pt}\selectfont Feltrin E.};

        % VERIFICATORE
        \node[anchor=north west, yshift=-1cm, align=left] (dati3) at (dati21.north west) {\fontsize{15pt}{15pt}\selectfont \textbf{Verificatore}};
        \draw[line width=4pt, lightcol] (dati3.south west) -- ([xshift=8cm]dati3.south west);
        \node[anchor=north west, align=left] (dati31) at (dati3.south west)
        {\fontsize{13pt}{13pt}\selectfont Ciriolo I.};

        % SEGRETARIO DI RIUNIONE
        \node[anchor=north west, yshift=-1cm, align=left] (dati4) at (dati31.north west) {\fontsize{15pt}{15pt}\selectfont \textbf{Segretario di Riunione}};
        \draw[line width=4pt, lightcol] (dati4.south west) -- ([xshift=8cm]dati4.south west);
        \node[anchor=north west, align=left] (dati41) at (dati4.south west)
        {\fontsize{13pt}{13pt}\selectfont Campese M.};

        % PARTECIPANTI
        %\node[anchor=north west, yshift=-1cm, align=left] (dati5) at (dati41.north west) {\fontsize{15pt}{15pt}\selectfont \textbf{Partecipanti}};
        %\draw[line width=4pt, lightcol] (dati5.south west) -- ([xshift=8cm]dati5.south west);
        %\node[anchor=north west, align=left] (dati51) at (dati5.south west)
        %{\fontsize{13pt}{13pt}\selectfont Campese M.};
        %\node[anchor=north west, yshift=-0.7cm, align=left] (dati52) at (dati5.south west)
        %{\fontsize{13pt}{13pt}\selectfont Ciriolo I.};
        %\node[anchor=north west, yshift=-1.4cm, align=left] (dati53) at (dati5.south west)
        %{\fontsize{13pt}{13pt}\selectfont Dugo A.};
        %\node[anchor=north west, yshift=-2.1cm, align=left] (dati54) at (dati5.south west)
        %{\fontsize{13pt}{13pt}\selectfont Michelon R.};
        
        % UNIPD - SWE
        \node [xshift=4.4cm, yshift=2.3cm, draw, secondarycolor, text=white] (uni) at (current page.south west) {\fontsize{20pt}{20pt} \selectfont Università di Padova};
        \node [xshift=0.65cm, yshift=0.7cm, draw, secondarycolor, text=white, below=of uni] (corso) {\fontsize{20pt}{20pt}\selectfont Ingegneria del Software};

        % FIRMA
        \draw[line width=4pt, lightcol] ([xshift=-1.2cm, yshift=1.8cm]current page.south east) -- ([xshift=-8cm, yshift=1.8cm]current page.south east);
        \node [xshift=-4.8cm, yshift=2.45cm] (logo) at (current page.south east) {\includegraphics[width=6cm]{firme/emanuele.png}};
        \node[anchor=north west, xshift=12.9cm, yshift=1.45cm, align=left] at (current page.south west)
        {\fontsize{13pt}{13pt}\selectfont Il Responsabile: Feltrin E.};
        
    \end{tikzpicture}
\end{titlepage}

% INIZIO PAGINE
% use to vertically center content
% credits to: http://tex.stackexchange.com/questions/7219/how-to-vertically-center-two-images-next-to-each-other
\newcommand{\vcenteredinclude}[1]{\begingroup
\setbox0=\hbox{\includegraphics{#1}}%
\parbox{\wd0}{\box0}\endgroup}

% use to vertically center content
% credits to: http://tex.stackexchange.com/questions/7219/how-to-vertically-center-two-images-next-to-each-other
\newcommand*{\vcenteredhbox}[1]{\begingroup
\setbox0=\hbox{#1}\parbox{\wd0}{\box0}\endgroup}

\newcommand{\heading}[1] {
	\vspace{15pt}
	{\bf\LARGE\color{secondarycolor}\uppercase{#1}}\\[-4pt]
	{\color{primarycolor}\rule{0.1\textwidth}{2pt}}\vspace{2pt}
}

%---------------------------------------------------------
% New 
%----------------------------------------------------------
\newcommand{\titlebox}[3]
{\fcolorbox{#1}{#2}{\begin{minipage}[c][4.5cm][c]{\linewidth}%
\begin{center}\large\color{white} #3 %
\end{center}\end{minipage}\\[14pt]
\vspace{-12pt}
}
}

\newcommand{\bigfont}[1]{%
{\bf\huge\uppercase{#1} } \\[4pt]%
\rule{0.1\textwidth}{1.25pt} \\[4pt]%
}

\newcommand{\titletext}[1]{%
#1 \\[4pt] %
\rule{0.1\textwidth}{1.25pt} \\[4pt]%
}
           



\setlength{\parindent}{0mm}

\setlist[itemize]{label=\color{primarycolor}\textbullet}

%============================================================================%
\columnratio{0.31}
\setlength{\columnsep}{2.2em}
\setlength{\columnseprule}{4pt}
\colseprulecolor{lightcol}
\begin{paracol}{2}

\heading{Intestazione}

\textbf{Data} \\
2023-10-27\\

\textbf{Ora Inizio} \\
10:00\\

\textbf{Ora Fine} \\
12:00\\

\textbf{Luogo} \\
Piattaforma Discord

\vspace{12.6em}

\heading{Partecipanti}

\textbf{Interni} \\
Bresolin G.\\
Campese M.\\
Ciriolo I.\\
Dugo A.\\
Feltrin E.\\
Michelon R.\\
Orlandi G.\\


\switchcolumn

%---------------------------------------------------------------------------------------


\heading{Revisione delle Azioni}
\begin{enumerate}
    \item Presa visione "Norme di progetto" preliminari stilate riguardanti la documentazione e della Lettera di presentazione da presentare alla candidatura;
    \item Revisione dei template design per i verbali interni ed esterni e controllo dell'omogeneità dei documenti finora redatti con successive modifiche correttive;
    \item Controllo e messa a punto della presentazione per il Diario di Bordo del giorno 2023-10-30;
    \item Riesame e approvazione resoconto dei capitolati.
\end{enumerate}

\vspace{14.7em}

\heading{Ordine del Giorno}
\begin{enumerate}
    \item Assegnazione ruoli riunione;
    \item Discussione preventivo e costo orario; 
    \item Ridefinizione verbali esterni;
    \item Preparazione della candidatura.
\end{enumerate}

\newpage

\switchcolumn
\newpage

\heading{Discussione}\\
\textbf{Sintesi degli argomenti\\discussi}

\vspace{13.6cm}

\heading{Decisioni}\\
\textbf{Decisioni prese durante\\la discussione}

\switchcolumn
\begin{enumerate}
     \item Assegnazione ruoli riunione:
    \begin{itemize}
        \item Responsabile: Feltrin E.;
        \item Segretario di riunione: Campese M.;
        \item Responsabile della revisione: Ciriolo I.
    \end{itemize}
    \item In seguito ad una attenta fase di analisi delle preferenze generali dei membri del gruppo sono state individuate le ore di lavoro assegnate ad ognuno di essi per ciascun ruolo fino alla consegna stimata del progetto. I dati vengono formalizzati nel documento da consegnare alla candidatura;
    \item Durante la riunione si è discusso dei verbali redatti a seguito delle riunioni con gli esponenti di Zucchetti, AzzurroDigitale ed Ergon; si è arrivati alla conclusione di dover richiedere la validazione dei verbali da parte delle aziende perché questi abbiano valore ufficiale;
    \item Riesame delle risorse necessarie per la presentazione della candidatura e relativo stato di avanzamento. Emerge l'urgenza di inizializzare e configurare una pagina ufficiale \textit{GitHub.io} per la presentazione di:
    \begin{itemize}
        \item Lettera di presentazione;
        \item Preventivo costi;
        \item Resoconto dei capitolati presentati il giorno 2023-10-17.
    \end{itemize}
\end{enumerate}

\vspace{7em}

\begin{enumerate}
    \item Decisione ruoli iniziali per il progetto da attuare in seguito alla assegnazione del capitolato scelto;   
    \item Necessità di validazione verbali esterni da parte delle aziende proponenti;
    \item Consegna del prodotto finito relativo al capitolato C1, “Knowledge
Management AI”, proposto dall’azienda AzzurroDigitale, stimata entro il \textbf{2024-04-08};
\item Stima costo finale del progetto pari a \textbf{12.845,00 €}.
    
\end{enumerate}

\end{paracol}

\newpage

\heading{Azioni da Intraprendere}

{\renewcommand{\arraystretch}{1.5}
\begin{tabularx}{\textwidth}{X|c|c|c}
\textbf{Azione} & \textbf{Incaricato} & \textbf{Revisore} & \textbf{Scadenza} \\
\hline
Stesura verbale interno $2023-10-27$ & Campese M. & Ciriolo I. & $2023-10-28$ \\
\hline
Stesura preventivo costi  & \quantities{Bresolin G. \\ Ciriolo I.\\ Dugo A.} & Team & $2023-10-29$ \\
\hline
Configurazione iniziale pagina \textit{GitHub.io} & \quantities{Dugo A. \\ Michelon R.} & Team & $2023-10-30$ \\
\hline
Richiesta validazione verbale $2023-10-23$ & Ciriolo I. & Team & $2023-10-31$ \\
\hline
Richiesta validazione verbale $2023-10-25$ & Bresolin G. & Team & $2023-10-31$ \\
\hline
Richiesta validazione verbale $2023-10-26$ & Dugo A. & Team & $2023-10-31$ \\

\end{tabularx}}

\vspace{3em}

\heading{Altro}

\textbf{Prossima Riunione}\\
Non è stata fissata una data per una nuova riunione.

\end{document}