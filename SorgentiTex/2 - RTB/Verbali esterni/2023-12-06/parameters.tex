% Parametri che modificano il file main.tex
% Le uniche parti da cambiare su main.tex sono:
% - vari \vspace tra sezioni
% - tabella azioni da intraprendere
% - sezione altro

\def\data{2023-12-06}
\def\oraInizio{15:00}
\def\oraFine{16:00}
\def\luogo{Google Meet}

\def\tipoVerb{Esterno} % Interno - Esterno

\def\nomeResp{Dugo A.} % Cognome N.
\def\nomeVer{Bresolin G.} % Cognome N.
\def\nomeSegr{Dugo A.} % Cognome N.

\def\nomeAzienda{Azzurro Digitale}
\def\firmaAzienda{azzurrodigitale.png}
\def\firmaResp{alberto.png} % nome Responsabile

\def\listaPartInt{
Bresolin G.,
Campese M.,
Ciriolo I.,
Dugo A.,
Feltrin E.,
Michelon R.,
Orlandi G.
}

\def\listaPartEst{
Bendotti E.,
Davanzo C.
}

% Se nessuna revisione: \def\listaRevisioneAzioni {x}
\def\listaRevisioneAzioni {x}

\def\listaOrdineGiorno {
{Presentazione del PoC definitivo.}
}

\def\listaDiscussioneInterna {
{Discussione 1;},
{Discussione 2;},
}

\newcommand{\domris}[2]{\textbf{#1}\\#2}


\def\listaDiscussioneEsterna {
\domris
{PoC proposto da SWEetCode
}{
    Il PoC, creato dai programmatori (Alberto D., Emanuele F., Riccardo M.) nel secondo sprint e presentato da Riccardo M., incorpora le conoscenze acquisite nei PoC precedenti, citati nel verbale esterno (2023-11-29).
    Il PoC presentato implementa le richieste effettuate dal proponente nella precedente riunione:
    \begin{itemize}
        \item Refactoring del codice proposto dal team;
        \item Presenza di un pattern architetturale;
        \item Sostituzione di \textit{Streamlit} con \textit{React};
        \item Ristrutturazione interfaccia utente.
    \end{itemize} 
    Il sistema proposto è composto da un frontend sviluppato in \textit{React} e un backend in \textit{Python}. Il dialogo tra queste parti avviene tramite API implementate attraverso il framework \textit{Flask}.
    A differenza dei PoC presentati precedentemente, viene utilizzato \textit{Pinecone} come database vettoriale, poichè fornisce uno spazio remoto gratuito di archiviazione dei vettori.
    Il team ha inoltre scelto il servizio \textit{S3} di \textit{Amazon Web Services (AWS)} come soluzione di archiviazione dei documenti PDF caricati dall'utente. Il collegamento tra questo servizio e il backend implementato è stato effettuato tramite l'uso di \textit{Boto3}.
    Le scelte tecnologiche incluse in questo PoC risultano essere definitive, poiché ritenute dal team valide e solide.
    I membri dell'azienda proponente, Davanzo C. e Bendotti E., hanno però espresso delle perplessità riguardo le risposte che il chatbot fornisce, non sempre pertinenti e corrette.
    Anche attraverso prove fatte durante la riunione, il team ha individuato alcune migliorie, che verranno apportate in seguito durante il prossimo sprint.
    I responsabili Davanzo C. e Bendotti E. hanno quindi approvato il PoC proposto con soddisfazione.
    }
}
% Se nessuna decisione: \def\listaDecisioni {x}
\def\listaDecisioni {
{Approvazione del PoC presentato.}
}